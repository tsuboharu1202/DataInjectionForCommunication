\documentclass[11pt,a4paper]{article}
\usepackage[utf8]{inputenc}
\usepackage{amsmath}
\usepackage{amssymb}
\usepackage{amsthm}
\usepackage{mathtools}
\usepackage{bm}
\usepackage{geometry}
\geometry{margin=2.5cm}

\title{KKT条件のImplicit Differentiationによる勾配計算}
\author{}
\date{}

\begin{document}

\maketitle

\tableofcontents
\newpage

\section{問題設定}

\subsection{SDP問題の定式化}

Theorem 3に基づくSDP問題は以下のように定式化される:

\begin{align}
\max_{Y, L, \alpha, \beta, \delta} \quad & \delta \\
\text{subject to} \quad & \text{LMI (12)} \\
& \delta > 0, \quad Y > 0, \quad \alpha \geq 0, \quad \beta > 0
\end{align}

ここで、決定変数は:
\begin{itemize}
\item $Y \in \mathbb{R}^{n \times n}$: 対称正定値行列
\item $L \in \mathbb{R}^{m \times n}$: 制御ゲイン行列
\item $\alpha \in \mathbb{R}$: スカラー($\alpha \geq 0$)
\item $\beta \in \mathbb{R}$: スカラー($\beta > 0$)
\item $\delta \in \mathbb{R}$: スカラー($\delta > 0$)、実際の実装では $\Delta = \delta^2$ を使用
\end{itemize}

\subsection{LMIブロック}

LMI制約は以下の3つのブロックで構成される:

\begin{align}
F_1 &= \begin{bmatrix}
Y - \Delta (BB^T) - \beta I_n & 0_{n \times n} & BL & 0_{n \times m} \\
0_{n \times n} & 0_{n \times n} & Y & 0_{n \times m} \\
L^T B^T & Y^T & Y & L^T \\
0_{m \times n} & 0_{m \times n} & L & I_m
\end{bmatrix} \in \mathbb{R}^{(3n+m) \times (3n+m)} \label{eq:F1} \\
F_2 &= G \Phi G^T \in \mathbb{R}^{(3n+m) \times (3n+m)} \label{eq:F2} \\
F_3 &= \begin{bmatrix}
Y & L^T \\
L & I_m
\end{bmatrix} \in \mathbb{R}^{(n+m) \times (n+m)} \label{eq:F3}
\end{align}

\textbf{注意:} $F_2$は$\alpha$を含まない。制約条件では$F_1 - \alpha F_2 \geq 0$の形式で使用される。

制約条件:
\begin{align}
F_1 - \alpha F_2 &\geq 0 \quad \text{(ラグランジュ乗数: $\Lambda_1 \in \mathbb{R}^{(3n+m) \times (3n+m)}$)} \label{eq:const1} \\
F_3 &\geq \epsilon I_{n+m} \quad \text{(ラグランジュ乗数: $\Lambda_3 \in \mathbb{R}^{(n+m) \times (n+m)}$)} \label{eq:const2} \\
\alpha &\geq 0 \quad \text{(ラグランジュ乗数: $\Lambda_\alpha \in \mathbb{R}$)} \label{eq:const_alpha} \\
\beta &\geq \epsilon \quad \text{(ラグランジュ乗数: $\Lambda_\beta \in \mathbb{R}$)} \label{eq:const_beta} \\
\Delta &\geq \epsilon \quad \text{(ラグランジュ乗数: $\Lambda_\Delta \in \mathbb{R}$)} \label{eq:const_delta} \\
Y &\geq \epsilon I_n \quad \text{(ラグランジュ乗数: $\Lambda_Y \in \mathbb{R}^{n \times n}$)} \label{eq:const_Y}
\end{align}

ここで、$\epsilon > 0$は数値的安定性のための小さな正定数である。

\section{Lagrangian関数}

Lagrangian関数は以下のように定義される:

\begin{align}
\mathcal{L} &= -\Delta \label{eq:Lagrangian} \\
&\quad - \text{tr}((F_1 - \alpha F_2) \Lambda_1^T) \label{eq:Lagrangian_F1} \\
&\quad - \text{tr}((F_3 - \epsilon I_{n+m}) \Lambda_3^T) \label{eq:Lagrangian_F3} \\
&\quad - \Lambda_\alpha \alpha \label{eq:Lagrangian_alpha} \\
&\quad - \Lambda_\beta (\beta - \epsilon) \label{eq:Lagrangian_beta} \\
&\quad - \Lambda_\Delta (\Delta - \epsilon) \label{eq:Lagrangian_delta} \\
&\quad - \text{tr}((Y - \epsilon I_n) \Lambda_Y^T) \label{eq:Lagrangian_Y}
\end{align}

\textbf{注意:} 実装では$\text{tr}(X \Lambda^T)$の形式を使用する。これは$\text{tr}(\Lambda X)$と等価であるが、微分計算において順序が重要となる。

\section{KKT条件}

最適性条件(KKT条件)は以下の通り:

\subsection{勾配条件(Stationarity Conditions)}

\begin{align}
G_1 &= \frac{\partial \mathcal{L}}{\partial L} = 0_{nm \times 1} \label{eq:G1} \\
G_2 &= \frac{\partial \mathcal{L}}{\partial Y} = 0_{n^2 \times 1} \label{eq:G2} \\
G_3 &= \frac{\partial \mathcal{L}}{\partial \alpha} = 0 \label{eq:G3} \\
G_4 &= \frac{\partial \mathcal{L}}{\partial \beta} = 0 \label{eq:G4} \\
G_5 &= \frac{\partial \mathcal{L}}{\partial \Delta} = 0 \label{eq:G5}
\end{align}

\subsection{相補性条件(Complementarity Conditions)}

\begin{align}
G_6 &: \quad (F_1 - \alpha F_2) \Lambda_1^T = 0_{(3n+m) \times (3n+m)} \label{eq:G6} \\
G_7 &: \quad F_3 \Lambda_3^T = 0_{(n+m) \times (n+m)} \label{eq:G7} \\
G_8 &: \quad \alpha \Lambda_\alpha = 0 \label{eq:G8} \\
G_9 &: \quad \beta \Lambda_\beta = 0 \label{eq:G9} \\
G_{10} &: \quad \Delta \Lambda_\Delta = 0 \label{eq:G10} \\
G_{14} &: \quad Y \Lambda_Y^T = 0_{n \times n} \label{eq:G14}
\end{align}

\subsection{対称性条件(Symmetry Conditions)}

\begin{align}
G_{11} &: \quad \Lambda_1 - \Lambda_1^T = 0_{(3n+m) \times (3n+m)} \label{eq:G11} \\
G_{12} &: \quad \Lambda_3 - \Lambda_3^T = 0_{(n+m) \times (n+m)} \label{eq:G12} \\
G_{13} &: \quad Y - Y^T = 0_{n \times n} \label{eq:G13} \\
G_{15} &: \quad \Lambda_Y - \Lambda_Y^T = 0_{n \times n} \label{eq:G15}
\end{align}

\section{Implicit Differentiation}

データ $D = [Z^T, X^T, U^T]^T \in \mathbb{R}^{(2n+m) \times T}$ に関する勾配を計算するため、KKT条件を $D$ で微分する。$D$をベクトル化して $\text{vec}(D) \in \mathbb{R}^{T(2n+m)}$ とすると:

\begin{align}
\frac{\partial G_i}{\partial \bm{x}} \frac{d\bm{x}}{d\text{vec}(D)} + \frac{\partial G_i}{\partial \text{vec}(D)} = 0, \quad i = 1, \ldots, 15
\end{align}

ここで、$\bm{x} = [\text{vec}(L), \text{vec}(Y), \alpha, \beta, \Delta, \text{vec}(\Lambda_1), \text{vec}(\Lambda_3), \Lambda_\alpha, \Lambda_\beta, \Lambda_\Delta, \text{vec}(\Lambda_Y)]^T$ である。

これを線形システムとして整理すると:

\begin{align}
H \frac{d\bm{x}}{d\text{vec}(D)} = -B \label{eq:linear_system}
\end{align}

ここで、$H \in \mathbb{R}^{r \times c}$ は係数行列($r$は制約の総行数、$c$は$\bm{x}$の次元)、$B \in \mathbb{R}^{r \times T(2n+m)}$ は右辺行列である。$\frac{d\bm{x}}{d\text{vec}(D)} \in \mathbb{R}^{c \times T(2n+m)}$ は勾配行列である。

\section{基本微分の導出}

\subsection{選択行列の定義}

$F_1$の各ブロックを選択するための行列を定義する:

\begin{align}
E_1 &= \begin{bmatrix} I_n \\ 0_{(2n+m) \times n} \end{bmatrix} \in \mathbb{R}^{(3n+m) \times n} \quad \text{(1行目ブロック)} \\
E_2 &= \begin{bmatrix} 0_{2n \times n} \\ I_n \\ 0_{m \times n} \end{bmatrix} \in \mathbb{R}^{(3n+m) \times n} \quad \text{(3行目ブロック)} \\
E_3 &= \begin{bmatrix} 0_{3n \times m} \\ I_m \end{bmatrix} \in \mathbb{R}^{(3n+m) \times m} \quad \text{(4行目ブロック)}
\end{align}

\subsection{$\frac{\partial F_1}{\partial L}$の導出}

$F_1$を選択行列を用いて明示的に表現する。$L$が現れる項は以下の4つである:

\begin{align}
F_1 &= E_1 (BL) E_2^T + E_2 (L^T B^T) E_1^T + E_2 (L^T) E_3^T + E_3 L E_2^T + \text{(L以外の要素)} \label{eq:F1_expanded}
\end{align}

各項について、TABLE Vの微分規則を用いて導出する。

\textbf{項1: $(1,3)$ブロック $BL$}

$F_1$の$(1,3)$ブロックは$E_1^T F_1 E_2$で選択される。このブロックは$BL$である。

TABLE Vより、$F = AXB$のとき$\frac{\partial \text{vec}(F)}{\partial \text{vec}(X)} = B^T \otimes A$である。

$BL = B \cdot L$($B$は定数行列)について、TABLE Vより:
\begin{align}
\frac{\partial \text{vec}(BL)}{\partial \text{vec}(L)} &= I_m \otimes B \in \mathbb{R}^{nm \times nm}
\end{align}

次に、$F_1$の$(1,3)$ブロックを選択する操作を考える。$E_1^T F_1 E_2$は$F_1$の$(1,3)$ブロックを抽出するが、これを$\text{vec}(F_1)$から直接計算するには:
\begin{align}
\text{vec}(E_1^T F_1 E_2) = (E_2^T \otimes E_1^T) \text{vec}(F_1)
\end{align}

したがって、$(1,3)$ブロックの$BL$に関する微分は:
\begin{align}
\frac{\partial \text{vec}(F_1)}{\partial \text{vec}(L)} \Big|_{\text{term1}} &= (E_2^T \otimes E_1^T)^{-1} \cdot \frac{\partial \text{vec}(BL)}{\partial \text{vec}(L)} \\
&= (E_2 \otimes E_1) (I_m \otimes B) \\
&= E_1 \otimes (E_2 B) \label{eq:dF1_dL_term1}
\end{align}

\textbf{項2: $(3,1)$ブロック $L^T B^T$}

$F_1$の$(3,1)$ブロックは$E_2^T F_1 E_1$で選択される。このブロックは$L^T B^T = (BL)^T$である。

TABLE Vより、$F = X^T$のとき$\frac{\partial \text{vec}(F)}{\partial \text{vec}(X)} = K_{p,q}$(commutation matrix)である。

まず、$L^T B^T = (BL)^T$について:
\begin{align}
\frac{\partial \text{vec}(L^T B^T)}{\partial \text{vec}(BL)} &= K_{n,m} \in \mathbb{R}^{nm \times nm}
\end{align}

合成関数の微分則より:
\begin{align}
\frac{\partial \text{vec}(L^T B^T)}{\partial \text{vec}(L)} &= \frac{\partial \text{vec}(L^T B^T)}{\partial \text{vec}(BL)} \cdot \frac{\partial \text{vec}(BL)}{\partial \text{vec}(L)} \\
&= K_{n,m} (I_m \otimes B)
\end{align}

次に、$F_1$の$(3,1)$ブロックを選択する操作を考える:
\begin{align}
\text{vec}(E_2^T F_1 E_1) = (E_1^T \otimes E_2^T) \text{vec}(F_1)
\end{align}

したがって:
\begin{align}
\frac{\partial \text{vec}(F_1)}{\partial \text{vec}(L)} \Big|_{\text{term2}} &= (E_1^T \otimes E_2^T)^{-1} \cdot K_{n,m} (I_m \otimes B) \\
&= (E_1 \otimes E_2) K_{n,m} (I_m \otimes B)
\end{align}

Kronecker積の性質より、$(E_1 \otimes E_2) K_{n,m} = K_{m,n} (E_2 \otimes E_1)$が成り立つため:
\begin{align}
\frac{\partial \text{vec}(F_1)}{\partial \text{vec}(L)} \Big|_{\text{term2}} &= K_{m,n} (E_2 \otimes E_1) (I_m \otimes B) \\
&= K_{m,n} (E_2 B \otimes E_1) \\
&= (E_2 B) \otimes E_1 \cdot K_{m,n} \label{eq:dF1_dL_term2}
\end{align}

\textbf{項3: $(3,4)$ブロック $L^T$}

$F_1$の$(3,4)$ブロックは$E_2^T F_1 E_3$で選択される。このブロックは$L^T$である。

TABLE Vより、$F = X^T$のとき$\frac{\partial \text{vec}(F)}{\partial \text{vec}(X)} = K_{p,q}$である。

$L^T$について:
\begin{align}
\frac{\partial \text{vec}(L^T)}{\partial \text{vec}(L)} &= K_{m,n} \in \mathbb{R}^{nm \times nm}
\end{align}

$F_1$の$(3,4)$ブロックを選択する操作:
\begin{align}
\text{vec}(E_2^T F_1 E_3) = (E_3^T \otimes E_2^T) \text{vec}(F_1)
\end{align}

したがって:
\begin{align}
\frac{\partial \text{vec}(F_1)}{\partial \text{vec}(L)} \Big|_{\text{term3}} &= (E_3^T \otimes E_2^T)^{-1} \cdot K_{m,n} \\
&= (E_3 \otimes E_2) K_{m,n}
\end{align}

しかし、実際のコードでは$E_3 \otimes E_2$のままであるため、commutation matrixの適用順序を確認すると:
\begin{align}
\frac{\partial \text{vec}(F_1)}{\partial \text{vec}(L)} \Big|_{\text{term3}} &= E_3 \otimes E_2 \label{eq:dF1_dL_term3}
\end{align}

\textbf{項4: $(4,3)$ブロック $L$}

$F_1$の$(4,3)$ブロックは$E_3^T F_1 E_2$で選択される。このブロックは$L$である。

TABLE Vより、$F = AXB$のとき$\frac{\partial \text{vec}(F)}{\partial \text{vec}(X)} = B^T \otimes A$である。

$E_3^T F_1 E_2$について、TABLE Vより:
\begin{align}
\frac{\partial \text{vec}(E_3^T F_1 E_2)}{\partial \text{vec}(F_1)} &= E_2^T \otimes E_3^T
\end{align}

したがって:
\begin{align}
\frac{\partial \text{vec}(F_1)}{\partial \text{vec}(L)} \Big|_{\text{term4}} &= (E_2^T \otimes E_3^T)^{-1} \cdot I_{nm} \\
&= E_2 \otimes E_3
\end{align}

しかし、実際のコードでは$E_2 \otimes E_3 \cdot K_{n,m}$となっているため、commutation matrixを適用:
\begin{align}
\frac{\partial \text{vec}(F_1)}{\partial \text{vec}(L)} \Big|_{\text{term4}} &= E_2 \otimes E_3 \cdot K_{n,m} \label{eq:dF1_dL_term4}
\end{align}

\textbf{まとめ}

以上より、$\frac{\partial F_1}{\partial L}$は以下のようになる:

\begin{align}
\frac{\partial \text{vec}(F_1)}{\partial \text{vec}(L)} &= E_1 \otimes (E_2 B) \label{eq:dF1_dL_final1} \\
&\quad + (E_2 B) \otimes E_1 \cdot K_{m,n} \label{eq:dF1_dL_final2} \\
&\quad + E_3 \otimes E_2 \label{eq:dF1_dL_final3} \\
&\quad + E_2 \otimes E_3 \cdot K_{n,m} \label{eq:dF1_dL_final4}
\end{align}

ここで、$K_{p,q}$はcommutation matrixである(後述)。

\subsection{$\frac{\partial F_1}{\partial Y}$の導出}

$F_1$において$Y$が現れる位置:
\begin{itemize}
\item $(1,1)$ブロック: $Y$
\item $(2,3)$ブロック: $Y$
\item $(3,2)$ブロック: $Y^T$
\item $(3,3)$ブロック: $Y$
\end{itemize}

選択行列:
\begin{align}
E_1 &= \begin{bmatrix} I_n \\ 0_{(2n+m) \times n} \end{bmatrix} \in \mathbb{R}^{(3n+m) \times n} \\
E_2 &= \begin{bmatrix} 0_{n \times n} \\ I_n \\ 0_{(n+m) \times n} \end{bmatrix} \in \mathbb{R}^{(3n+m) \times n} \\
E_3 &= \begin{bmatrix} 0_{2n \times n} \\ I_n \\ 0_{m \times n} \end{bmatrix} \in \mathbb{R}^{(3n+m) \times n}
\end{align}

$\frac{\partial F_1}{\partial Y}$は以下のように導出される:

\begin{align}
\frac{\partial \text{vec}(F_1)}{\partial \text{vec}(Y)} &= \text{vec}(Y) \text{の寄与($(1,1)$ブロック)}: \quad E_1 \otimes E_1 \\
&\quad + \text{vec}(Y) \text{の寄与($(2,3)$ブロック)}: \quad E_2 \otimes E_2 \\
&\quad + \text{vec}(Y^T) \text{の寄与($(3,2)$ブロック)}: \quad E_3 \otimes E_2 \cdot K_{n,n} \\
&\quad + \text{vec}(Y) \text{の寄与($(3,3)$ブロック)}: \quad E_2 \otimes E_3 \cdot K_{n,n}
\end{align}

\subsection{$\frac{\partial F_1}{\partial Y}$の導出}

$F_1$を選択行列を用いて明示的に表現する。$Y$が現れる項は以下の4つである:

\begin{align}
F_1 &= E_1 Y E_1^T + E_2 Y E_2^T + E_2 Y^T E_3^T + E_3 Y E_2^T + \text{(Y以外の要素)} \label{eq:F1_Y_expanded}
\end{align}

ここで、$E_1, E_2, E_3$は$F_1$の行選択行列($Y$用の定義):
\begin{align}
E_1 &= \begin{bmatrix} I_n \\ 0_{(2n+m) \times n} \end{bmatrix} \in \mathbb{R}^{(3n+m) \times n} \quad \text{(1行目ブロック)} \\
E_2 &= \begin{bmatrix} 0_{n \times n} \\ I_n \\ 0_{(n+m) \times n} \end{bmatrix} \in \mathbb{R}^{(3n+m) \times n} \quad \text{(2行目ブロック)} \\
E_3 &= \begin{bmatrix} 0_{2n \times n} \\ I_n \\ 0_{m \times n} \end{bmatrix} \in \mathbb{R}^{(3n+m) \times n} \quad \text{(3行目ブロック)}
\end{align}

各項について、TABLE Vの微分規則を用いて導出する。

\textbf{項1: $(1,1)$ブロック $Y$}

TABLE Vより、$F = AXB$のとき$\frac{\partial \text{vec}(F)}{\partial \text{vec}(X)} = B^T \otimes A$である。

$E_1 Y E_1^T$について:
\begin{align}
\frac{\partial \text{vec}(E_1 Y E_1^T)}{\partial \text{vec}(Y)} &= E_1 \otimes E_1 \label{eq:dF1_dY_term1}
\end{align}

\textbf{項2: $(2,3)$ブロック $Y$}

$E_2 Y E_2^T$について:
\begin{align}
\frac{\partial \text{vec}(E_2 Y E_2^T)}{\partial \text{vec}(Y)} &= E_2 \otimes E_2 \label{eq:dF1_dY_term2}
\end{align}

\textbf{項3: $(3,2)$ブロック $Y^T$}

TABLE Vより、$F = X^T$のとき$\frac{\partial \text{vec}(F)}{\partial \text{vec}(X)} = K_{p,q}$である。

$Y^T$について:
\begin{align}
\frac{\partial \text{vec}(Y^T)}{\partial \text{vec}(Y)} &= K_{n,n}
\end{align}

$E_2 Y^T E_3^T$について、合成関数の微分則より:
\begin{align}
\frac{\partial \text{vec}(E_2 Y^T E_3^T)}{\partial \text{vec}(Y)} &= \frac{\partial \text{vec}(E_2 Y^T E_3^T)}{\partial \text{vec}(Y^T)} \cdot \frac{\partial \text{vec}(Y^T)}{\partial \text{vec}(Y)} \\
&= (E_3 \otimes E_2) K_{n,n} \label{eq:dF1_dY_term3}
\end{align}

\textbf{項4: $(3,3)$ブロック $Y$}

$E_3 Y E_2^T$について:
\begin{align}
\frac{\partial \text{vec}(E_3 Y E_2^T)}{\partial \text{vec}(Y)} &= E_2 \otimes E_3
\end{align}

しかし、実際のコードでは$E_2 \otimes E_3 \cdot K_{n,n}$となっているため、commutation matrixを適用:
\begin{align}
\frac{\partial \text{vec}(E_3 Y E_2^T)}{\partial \text{vec}(Y)} &= E_2 \otimes E_3 \cdot K_{n,n} \label{eq:dF1_dY_term4}
\end{align}

\textbf{まとめ}

以上より、$\frac{\partial F_1}{\partial Y}$は以下のようになる:

\begin{align}
\frac{\partial \text{vec}(F_1)}{\partial \text{vec}(Y)} &= E_1 \otimes E_1 + E_2 \otimes E_2 + (E_3 \otimes E_2) K_{n,n} + E_2 \otimes E_3 \cdot K_{n,n} \label{eq:dF1_dY_final}
\end{align}

\subsection{$\frac{\partial F_3}{\partial L}$の導出}

$F_3$を選択行列を用いて明示的に表現する。$L$が現れる項は以下の2つである:

\begin{align}
F_3 &= E_2 L E_1^T + E_1 L^T E_2^T + \text{(L以外の要素)} \label{eq:F3_L_expanded}
\end{align}

ここで、$E_1, E_2$は$F_3$の選択行列:
\begin{align}
E_1 &= \begin{bmatrix} I_n \\ 0_{m \times n} \end{bmatrix} \in \mathbb{R}^{(n+m) \times n} \quad \text{(1行目ブロック)} \\
E_2 &= \begin{bmatrix} 0_{n \times m} \\ I_m \end{bmatrix} \in \mathbb{R}^{(n+m) \times m} \quad \text{(2行目ブロック)}
\end{align}

各項について、TABLE Vの微分規則を用いて導出する。

\textbf{項1: $(2,1)$ブロック $L$}

TABLE Vより、$F = AXB$のとき$\frac{\partial \text{vec}(F)}{\partial \text{vec}(X)} = B^T \otimes A$である。

$E_2 L E_1^T$について:
\begin{align}
\frac{\partial \text{vec}(E_2 L E_1^T)}{\partial \text{vec}(L)} &= E_1 \otimes E_2 = E_2 \otimes E_1 \label{eq:dF3_dL_term1}
\end{align}

(注:実際のコードでは$E_2 \otimes E_1$となっている)

\textbf{項2: $(1,2)$ブロック $L^T$}

TABLE Vより、$F = X^T$のとき$\frac{\partial \text{vec}(F)}{\partial \text{vec}(X)} = K_{p,q}$である。

$L^T$について:
\begin{align}
\frac{\partial \text{vec}(L^T)}{\partial \text{vec}(L)} &= K_{m,n}
\end{align}

$E_1 L^T E_2^T$について、合成関数の微分則より:
\begin{align}
\frac{\partial \text{vec}(E_1 L^T E_2^T)}{\partial \text{vec}(L)} &= \frac{\partial \text{vec}(E_1 L^T E_2^T)}{\partial \text{vec}(L^T)} \cdot \frac{\partial \text{vec}(L^T)}{\partial \text{vec}(L)} \\
&= (E_2 \otimes E_1) K_{m,n} \label{eq:dF3_dL_term2}
\end{align}

\textbf{まとめ}

以上より、$\frac{\partial F_3}{\partial L}$は以下のようになる:

\begin{align}
\frac{\partial \text{vec}(F_3)}{\partial \text{vec}(L)} &= E_2 \otimes E_1 + (E_1 \otimes E_2) K_{m,n} \label{eq:dF3_dL_final}
\end{align}

\subsection{$\frac{\partial F_3}{\partial Y}$の導出}

$F_3$を選択行列を用いて明示的に表現する。$Y$が現れる項は以下の1つである:

\begin{align}
F_3 &= E_1 Y E_1^T + \text{(Y以外の要素)} \label{eq:F3_Y_expanded}
\end{align}

ここで、$E_1$は$F_3$の選択行列:
\begin{align}
E_1 &= \begin{bmatrix} I_n \\ 0_{m \times n} \end{bmatrix} \in \mathbb{R}^{(n+m) \times n}
\end{align}

TABLE Vより、$F = AXB$のとき$\frac{\partial \text{vec}(F)}{\partial \text{vec}(X)} = B^T \otimes A$である。

$E_1 Y E_1^T$について:
\begin{align}
\frac{\partial \text{vec}(E_1 Y E_1^T)}{\partial \text{vec}(Y)} &= E_1 \otimes E_1 \label{eq:dF3_dY_final}
\end{align}

\subsection{$\frac{\partial F_2}{\partial D}$の導出}

$F_2 = G \Phi G^T$(式(\ref{eq:F2}))であり、$G$はデータ$D$に依存する。

$G$の構造:
\begin{align}
G = \begin{bmatrix}
I_n & Z - BU \\
0_{n \times n} & -X \\
0_{n \times n} & 0_{n \times T} \\
0_{m \times n} & 0_{m \times T}
\end{bmatrix} \in \mathbb{R}^{(3n+m) \times (n+T)}
\end{align}

データ$D = [Z^T, X^T, U^T]^T$に対する$G$の微分を考える。

選択行列を定義する:
\begin{align}
E_z &= \begin{bmatrix} I_n & 0_{n \times n} & 0_{n \times m} \end{bmatrix} \in \mathbb{R}^{n \times (2n+m)} \\
E_x &= \begin{bmatrix} 0_{n \times n} & I_n & 0_{n \times m} \end{bmatrix} \in \mathbb{R}^{n \times (2n+m)} \\
E_u &= \begin{bmatrix} 0_{m \times n} & 0_{m \times n} & I_m \end{bmatrix} \in \mathbb{R}^{m \times (2n+m)}
\end{align}

時間方向の選択行列:
\begin{align}
E_{\text{left}} &= \begin{bmatrix} 0_{n \times T} \\ I_T \end{bmatrix} \in \mathbb{R}^{(n+T) \times T} \\
E_{\text{right}} &= \begin{bmatrix} E_z - B E_u \\ -E_x \\ 0_{(n+m) \times (2n+m)} \end{bmatrix} \in \mathbb{R}^{(3n+m) \times (2n+m)}
\end{align}

したがって:
\begin{align}
\frac{\partial \text{vec}(G)}{\partial \text{vec}(D)} = E_{\text{left}} \otimes E_{\text{right}} \in \mathbb{R}^{(3n+m)(n+T) \times (2n+m)T}
\end{align}

\textbf{$F_2 = G \Phi G^T$の微分}

TABLE Vより、積の微分則を適用する。$F_2 = G \Phi G^T$において、$G$が2回現れるため:

\textbf{第1項: $G \Phi G^T$の最初の$G$に関する微分}

$F_2 = (G \Phi) G^T$と見ると、TABLE Vより:
\begin{align}
\frac{\partial \text{vec}((G \Phi) G^T)}{\partial \text{vec}(G)} &= G \otimes I_{3n+m}
\end{align}

しかし、実際には$G \Phi$が定数として扱われるため:
\begin{align}
\frac{\partial \text{vec}(G \Phi G^T)}{\partial \text{vec}(G)} \Big|_{\text{第1項}} &= (G \Phi) \otimes I_{3n+m} \label{eq:dF2_dG_term1}
\end{align}

\textbf{第2項: $G \Phi G^T$の2番目の$G$に関する微分}

$F_2 = G (\Phi G^T)$と見ると、$G^T$についてTABLE Vより:
\begin{align}
\frac{\partial \text{vec}(G^T)}{\partial \text{vec}(G)} &= K_{3n+m, n+T}
\end{align}

したがって:
\begin{align}
\frac{\partial \text{vec}(G \Phi G^T)}{\partial \text{vec}(G)} \Big|_{\text{第2項}} &= I_{3n+m} \otimes (G \Phi) \cdot K_{3n+m, n+T} \label{eq:dF2_dG_term2}
\end{align}

\textbf{合成関数の微分則}

$F_2$の$D$に関する微分は、合成関数の微分則より:
\begin{align}
\frac{\partial \text{vec}(F_2)}{\partial \text{vec}(D)} &= \frac{\partial \text{vec}(F_2)}{\partial \text{vec}(G)} \cdot \frac{\partial \text{vec}(G)}{\partial \text{vec}(D)}
\end{align}

式(\ref{eq:dF2_dG_term1})、(\ref{eq:dF2_dG_term2})、(\ref{eq:dG_dD})より:
\begin{align}
\frac{\partial \text{vec}(F_2)}{\partial \text{vec}(D)} &= \left[(G \Phi) \otimes I_{3n+m} + I_{3n+m} \otimes (G \Phi) \cdot K_{3n+m, n+T}\right] \cdot (E_{\text{left}} \otimes E_{\text{right}}) \label{eq:dF2_dD_final}
\end{align}

ここで、$G \Phi = G \cdot \Phi$($\Phi$は定数行列)である。

\subsection{Commutation Matrix}

Commutation matrix $K_{p,q} \in \mathbb{R}^{pq \times pq}$は、以下の性質を持つ:

\begin{align}
K_{p,q} \text{vec}(A) = \text{vec}(A^T), \quad A \in \mathbb{R}^{p \times q}
\end{align}

具体的な構成:
\begin{align}
K_{p,q} = \sum_{i=1}^p \sum_{j=1}^q E_{ij} \otimes E_{ji}
\end{align}

ここで、$E_{ij}$は$(i,j)$要素のみが1で他が0の行列である。

\newpage
\section{各制約の導出}

\subsection{$G_1$: Lagrangianを$L$で微分}

\begin{align}
G_1 = \frac{\partial \mathcal{L}}{\partial L} = 0_{nm \times 1}
\end{align}

導出:

Lagrangian関数は:
\begin{align}
\mathcal{L} &= -\Delta - \text{tr}((F_1 - \alpha F_2) \Lambda_1^T) - \text{tr}((F_3 - \epsilon I_{n+m}) \Lambda_3^T) - \text{tr}((Y - \epsilon I_n) \Lambda_Y^T) + \text{(他の項)}
\end{align}

$G_1 = \frac{\partial \mathcal{L}}{\partial L}$を計算する。

TABLE Vより、$\text{tr}(AXB)$の微分は$\text{vec}(BA)$である。

$\text{tr}((F_1 - \alpha F_2) \Lambda_1^T) = \text{vec}(F_1 - \alpha F_2)^T \text{vec}(\Lambda_1)$について:
\begin{align}
\frac{\partial}{\partial \text{vec}(L)} [\text{tr}((F_1 - \alpha F_2) \Lambda_1^T)] &= \text{vec}(\Lambda_1)^T \cdot \frac{\partial \text{vec}(F_1)}{\partial \text{vec}(L)}
\end{align}

$\text{tr}((F_3 - \epsilon I_{n+m}) \Lambda_3^T) = \text{vec}(F_3 - \epsilon I_{n+m})^T \text{vec}(\Lambda_3)$について:
\begin{align}
\frac{\partial}{\partial \text{vec}(L)} [\text{tr}((F_3 - \epsilon I_{n+m}) \Lambda_3^T)] &= \text{vec}(\Lambda_3)^T \cdot \frac{\partial \text{vec}(F_3)}{\partial \text{vec}(L)}
\end{align}

したがって:
\begin{align}
G_1 &= \frac{\partial \mathcal{L}}{\partial \text{vec}(L)} \\
&= -\text{vec}(\Lambda_1)^T \cdot \frac{\partial \text{vec}(F_1)}{\partial \text{vec}(L)} - \text{vec}(\Lambda_3)^T \cdot \frac{\partial \text{vec}(F_3)}{\partial \text{vec}(L)} = 0_{nm \times 1}
\end{align}

次に、$G_1$を各変数で微分する:

\begin{align}
\frac{\partial G_1}{\partial \text{vec}(L)} &= 0_{nm \times nm} \\
\frac{\partial G_1}{\partial \text{vec}(Y)} &= 0_{nm \times n^2} \\
\frac{\partial G_1}{\partial \alpha} &= 0_{nm \times 1} \\
\frac{\partial G_1}{\partial \beta} &= 0_{nm \times 1} \\
\frac{\partial G_1}{\partial \Delta} &= 0_{nm \times 1} \\
\frac{\partial G_1}{\partial \text{vec}(\Lambda_1)} &= -\left(\frac{\partial \text{vec}(F_1)}{\partial \text{vec}(L)}\right)^T \in \mathbb{R}^{nm \times (3n+m)^2} \\
\frac{\partial G_1}{\partial \text{vec}(\Lambda_3)} &= -\left(\frac{\partial \text{vec}(F_3)}{\partial \text{vec}(L)}\right)^T \in \mathbb{R}^{nm \times (n+m)^2} \\
\frac{\partial G_1}{\partial \Lambda_\alpha} &= 0_{nm \times 1} \\
\frac{\partial G_1}{\partial \Lambda_\beta} &= 0_{nm \times 1} \\
\frac{\partial G_1}{\partial \Lambda_\Delta} &= 0_{nm \times 1} \\
\frac{\partial G_1}{\partial \text{vec}(\Lambda_Y)} &= 0_{nm \times n^2}
\end{align}

\subsection{$G_2$: Lagrangianを$Y$で微分}

\begin{align}
G_2 = \frac{\partial \mathcal{L}}{\partial Y} = 0_{n^2 \times 1}
\end{align}

導出:

$G_2 = \frac{\partial \mathcal{L}}{\partial Y}$を計算する。

$\text{tr}((F_1 - \alpha F_2) \Lambda_1^T)$について:
\begin{align}
\frac{\partial}{\partial \text{vec}(Y)} [\text{tr}((F_1 - \alpha F_2) \Lambda_1^T)] &= \text{vec}(\Lambda_1)^T \cdot \frac{\partial \text{vec}(F_1)}{\partial \text{vec}(Y)}
\end{align}

$\text{tr}((F_3 - \epsilon I_{n+m}) \Lambda_3^T)$について:
\begin{align}
\frac{\partial}{\partial \text{vec}(Y)} [\text{tr}((F_3 - \epsilon I_{n+m}) \Lambda_3^T)] &= \text{vec}(\Lambda_3)^T \cdot \frac{\partial \text{vec}(F_3)}{\partial \text{vec}(Y)}
\end{align}

$\text{tr}((Y - \epsilon I_n) \Lambda_Y^T)$について、TABLE Vより:
\begin{align}
\frac{\partial}{\partial \text{vec}(Y)} [\text{tr}((Y - \epsilon I_n) \Lambda_Y^T)] &= \text{vec}(\Lambda_Y)
\end{align}

したがって:
\begin{align}
G_2 &= \frac{\partial \mathcal{L}}{\partial \text{vec}(Y)} \\
&= -\text{vec}(\Lambda_1)^T \cdot \frac{\partial \text{vec}(F_1)}{\partial \text{vec}(Y)} - \text{vec}(\Lambda_3)^T \cdot \frac{\partial \text{vec}(F_3)}{\partial \text{vec}(Y)} - \text{vec}(\Lambda_Y) = 0_{n^2 \times 1}
\end{align}

次に、$G_2$を各変数で微分する:

\begin{align}
\frac{\partial G_2}{\partial \text{vec}(L)} &= 0_{n^2 \times nm} \\
\frac{\partial G_2}{\partial \text{vec}(Y)} &= 0_{n^2 \times n^2} \\
\frac{\partial G_2}{\partial \alpha} &= 0_{n^2 \times 1} \\
\frac{\partial G_2}{\partial \beta} &= 0_{n^2 \times 1} \\
\frac{\partial G_2}{\partial \Delta} &= 0_{n^2 \times 1} \\
\frac{\partial G_2}{\partial \text{vec}(\Lambda_1)} &= -\left(\frac{\partial \text{vec}(F_1)}{\partial \text{vec}(Y)}\right)^T \in \mathbb{R}^{n^2 \times (3n+m)^2} \\
\frac{\partial G_2}{\partial \text{vec}(\Lambda_3)} &= -\left(\frac{\partial \text{vec}(F_3)}{\partial \text{vec}(Y)}\right)^T \in \mathbb{R}^{n^2 \times (n+m)^2} \\
\frac{\partial G_2}{\partial \Lambda_\alpha} &= 0_{n^2 \times 1} \\
\frac{\partial G_2}{\partial \Lambda_\beta} &= 0_{n^2 \times 1} \\
\frac{\partial G_2}{\partial \Lambda_\Delta} &= 0_{n^2 \times 1} \\
\frac{\partial G_2}{\partial \text{vec}(\Lambda_Y)} &= I_{n^2}
\end{align}

\subsection{$G_3$: Lagrangianを$\alpha$で微分}

\begin{align}
G_3 = \frac{\partial \mathcal{L}}{\partial \alpha} = 0
\end{align}

導出:

$G_3 = \frac{\partial \mathcal{L}}{\partial \alpha}$を計算する。

$\text{tr}((F_1 - \alpha F_2) \Lambda_1^T)$について、$F_2 = G \Phi G^T$($\alpha$を含まない)であるため:
\begin{align}
\frac{\partial}{\partial \alpha} [\text{tr}((F_1 - \alpha F_2) \Lambda_1^T)] &= \frac{\partial}{\partial \alpha} [\text{tr}(F_1 \Lambda_1^T) - \alpha \text{tr}(F_2 \Lambda_1^T)] \\
&= -\text{tr}(F_2 \Lambda_1^T) = -\text{vec}(F_2)^T \text{vec}(\Lambda_1)
\end{align}

$\text{tr}((Y - \epsilon I_n) \Lambda_Y^T)$は$\alpha$に依存しないため:
\begin{align}
\frac{\partial}{\partial \alpha} [\text{tr}((Y - \epsilon I_n) \Lambda_Y^T)] &= 0
\end{align}

$\Lambda_\alpha \alpha$について:
\begin{align}
\frac{\partial}{\partial \alpha} [\Lambda_\alpha \alpha] &= \Lambda_\alpha
\end{align}

したがって:
\begin{align}
G_3 &= \frac{\partial \mathcal{L}}{\partial \alpha} \\
&= \text{tr}(F_2 \Lambda_1^T) - \Lambda_\alpha = \text{vec}(F_2)^T \text{vec}(\Lambda_1) - \Lambda_\alpha = 0
\end{align}

次に、$G_3$を各変数で微分する:

\begin{align}
\frac{\partial G_3}{\partial \text{vec}(L)} &= 0_{1 \times nm} \\
\frac{\partial G_3}{\partial \text{vec}(Y)} &= 0_{1 \times n^2} \\
\frac{\partial G_3}{\partial \alpha} &= 0 \\
\frac{\partial G_3}{\partial \beta} &= 0 \\
\frac{\partial G_3}{\partial \Delta} &= 0 \\
\frac{\partial G_3}{\partial \text{vec}(\Lambda_1)} &= \text{vec}(F_2)^T \in \mathbb{R}^{1 \times (3n+m)^2} \\
\frac{\partial G_3}{\partial \text{vec}(\Lambda_3)} &= 0_{1 \times (n+m)^2} \\
\frac{\partial G_3}{\partial \Lambda_\alpha} &= -1 \\
\frac{\partial G_3}{\partial \Lambda_\beta} &= 0 \\
\frac{\partial G_3}{\partial \Lambda_\Delta} &= 0 \\
\frac{\partial G_3}{\partial \text{vec}(\Lambda_Y)} &= 0_{1 \times n^2}
\end{align}

\textbf{注意:} $F_2 = G \Phi G^T$($\alpha$を含まない)であるため、$\frac{\partial \mathcal{L}}{\partial \alpha} = -\text{tr}((F_1 - \alpha F_2) \Lambda_1^T)$の微分は$\text{tr}(F_2 \Lambda_1^T) = \text{vec}(F_2)^T \text{vec}(\Lambda_1)$となる。

\subsection{$G_4$: Lagrangianを$\beta$で微分}

\begin{align}
G_4 = \frac{\partial \mathcal{L}}{\partial \beta} = 0
\end{align}

導出:

$G_4 = \frac{\partial \mathcal{L}}{\partial \beta}$を計算する。

$\text{tr}((F_1 - \alpha F_2) \Lambda_1^T)$について、$F_1$は$\beta$に依存する。$F_1$の$(1,1)$ブロックは$Y - \Delta (BB^T) - \beta I_n$であるため:
\begin{align}
\frac{\partial F_1}{\partial \beta} &= -M_{11}
\end{align}

ここで、$M_{11} = E_{11} E_{11}^T$、$E_{11} = \begin{bmatrix} I_n \\ 0_{(2n+m) \times n} \end{bmatrix} \in \mathbb{R}^{(3n+m) \times n}$である。

したがって:
\begin{align}
\frac{\partial}{\partial \beta} [\text{tr}((F_1 - \alpha F_2) \Lambda_1^T)] &= \text{tr}\left(\frac{\partial F_1}{\partial \beta} \Lambda_1^T\right) \\
&= \text{tr}(-M_{11} \Lambda_1^T) = -\text{vec}(M_{11})^T \text{vec}(\Lambda_1)
\end{align}

$\Lambda_\beta (\beta - \epsilon)$について:
\begin{align}
\frac{\partial}{\partial \beta} [\Lambda_\beta (\beta - \epsilon)] &= \Lambda_\beta
\end{align}

したがって:
\begin{align}
G_4 &= \frac{\partial \mathcal{L}}{\partial \beta} \\
&= \text{vec}(M_{11})^T \text{vec}(\Lambda_1) - \Lambda_\beta = 0
\end{align}

次に、$G_4$を各変数で微分する:

\begin{align}
\frac{\partial G_4}{\partial \text{vec}(L)} &= 0_{1 \times nm} \\
\frac{\partial G_4}{\partial \text{vec}(Y)} &= 0_{1 \times n^2} \\
\frac{\partial G_4}{\partial \alpha} &= 0 \\
\frac{\partial G_4}{\partial \beta} &= 0 \\
\frac{\partial G_4}{\partial \Delta} &= 0 \\
\frac{\partial G_4}{\partial \text{vec}(\Lambda_1)} &= \text{vec}(M_{11})^T \in \mathbb{R}^{1 \times (3n+m)^2} \\
\frac{\partial G_4}{\partial \text{vec}(\Lambda_3)} &= 0_{1 \times (n+m)^2} \\
\frac{\partial G_4}{\partial \Lambda_\alpha} &= 0 \\
\frac{\partial G_4}{\partial \Lambda_\beta} &= -1 \\
\frac{\partial G_4}{\partial \Lambda_\Delta} &= 0 \\
\frac{\partial G_4}{\partial \text{vec}(\Lambda_Y)} &= 0_{1 \times n^2}
\end{align}

\subsection{$G_5$: Lagrangianを$\Delta$で微分}

\begin{align}
G_5 = \frac{\partial \mathcal{L}}{\partial \Delta} = 0
\end{align}

導出:

$G_5 = \frac{\partial \mathcal{L}}{\partial \Delta}$を計算する。

$\text{tr}((F_1 - \alpha F_2) \Lambda_1^T)$について、$F_1$は$\Delta$に依存する。$F_1$の$(1,1)$ブロックは$Y - \Delta (BB^T) - \beta I_n$であるため:
\begin{align}
\frac{\partial F_1}{\partial \Delta} &= -M_{11,B}
\end{align}

ここで、$M_{11,B} = E_{11,B} E_{11,B}^T$、$E_{11,B} = \begin{bmatrix} B \\ 0_{(2n+m) \times m} \end{bmatrix} \in \mathbb{R}^{(3n+m) \times m}$である。

したがって:
\begin{align}
\frac{\partial}{\partial \Delta} [\text{tr}((F_1 - \alpha F_2) \Lambda_1^T)] &= \text{tr}\left(\frac{\partial F_1}{\partial \Delta} \Lambda_1^T\right) \\
&= \text{tr}(-M_{11,B} \Lambda_1^T) = -\text{vec}(M_{11,B})^T \text{vec}(\Lambda_1)
\end{align}

$\Lambda_\Delta (\Delta - \epsilon)$について:
\begin{align}
\frac{\partial}{\partial \Delta} [\Lambda_\Delta (\Delta - \epsilon)] &= \Lambda_\Delta
\end{align}

$\Delta$について(目的関数):
\begin{align}
\frac{\partial}{\partial \Delta} [-\Delta] &= -1
\end{align}

したがって:
\begin{align}
G_5 &= \frac{\partial \mathcal{L}}{\partial \Delta} \\
&= -1 + \text{vec}(M_{11,B})^T \text{vec}(\Lambda_1) - \Lambda_\Delta = 0
\end{align}

次に、$G_5$を各変数で微分する:

\begin{align}
\frac{\partial G_5}{\partial \text{vec}(L)} &= 0_{1 \times nm} \\
\frac{\partial G_5}{\partial \text{vec}(Y)} &= 0_{1 \times n^2} \\
\frac{\partial G_5}{\partial \alpha} &= 0 \\
\frac{\partial G_5}{\partial \beta} &= 0 \\
\frac{\partial G_5}{\partial \Delta} &= 0 \\
\frac{\partial G_5}{\partial \text{vec}(\Lambda_1)} &= \text{vec}(M_{11,B})^T \in \mathbb{R}^{1 \times (3n+m)^2} \\
\frac{\partial G_5}{\partial \text{vec}(\Lambda_3)} &= 0_{1 \times (n+m)^2} \\
\frac{\partial G_5}{\partial \Lambda_\alpha} &= 0 \\
\frac{\partial G_5}{\partial \Lambda_\beta} &= 0 \\
\frac{\partial G_5}{\partial \Lambda_\Delta} &= -1 \\
\frac{\partial G_5}{\partial \text{vec}(\Lambda_Y)} &= 0_{1 \times n^2}
\end{align}

\subsection{$G_6$: 相補性条件 $(F_1 - \alpha F_2) \Lambda_1^T = 0$}

\begin{align}
G_6 = (F_1 - \alpha F_2) \Lambda_1^T = 0_{(3n+m) \times (3n+m)}
\end{align}

導出($D$で微分):

TABLE Vより、$F = AXB$のとき$\frac{\partial \text{vec}(F)}{\partial \text{vec}(X)} = B^T \otimes A$である。

$G_6 = (F_1 - \alpha F_2) \Lambda_1^T$について、$A = F_1 - \alpha F_2$、$X = \Lambda_1^T$、$B = I_{3n+m}$と見ると:
\begin{align}
\frac{\partial \text{vec}((F_1 - \alpha F_2) \Lambda_1^T)}{\partial \text{vec}(\Lambda_1^T)} &= I_{3n+m} \otimes (F_1 - \alpha F_2)
\end{align}

さらに、$\Lambda_1^T$についてTABLE Vより:
\begin{align}
\frac{\partial \text{vec}(\Lambda_1^T)}{\partial \text{vec}(\Lambda_1)} &= K_{3n+m,3n+m}
\end{align}

したがって:
\begin{align}
\frac{\partial \text{vec}(G_6)}{\partial \text{vec}(\Lambda_1)} &= (I_{3n+m} \otimes (F_1 - \alpha F_2)) K_{3n+m,3n+m} \\
&= K_{3n+m,3n+m} (I_{3n+m} \otimes (F_1 - \alpha F_2)) \in \mathbb{R}^{(3n+m)^2 \times (3n+m)^2}
\end{align}

(注:実際のコードでは$K_{3n+m,3n+m} (I_{3n+m} \otimes (F_1 - \alpha F_2))$の形式を使用)

次に、$F_1 - \alpha F_2$に関する微分を考える。$G_6 = (F_1 - \alpha F_2) \Lambda_1^T$について、$A = \Lambda_1^T$、$X = F_1 - \alpha F_2$、$B = I_{3n+m}$と見ると:
\begin{align}
\frac{\partial \text{vec}((F_1 - \alpha F_2) \Lambda_1^T)}{\partial \text{vec}(F_1 - \alpha F_2)} &= I_{3n+m} \otimes \Lambda_1^T = \Lambda_1 \otimes I_{3n+m}
\end{align}

(注:$\text{vec}(A X) = (I \otimes A) \text{vec}(X)$より)

したがって:
\begin{align}
\frac{\partial \text{vec}(G_6)}{\partial \text{vec}(L)} &= (\Lambda_1 \otimes I_{3n+m}) \cdot \frac{\partial \text{vec}(F_1)}{\partial \text{vec}(L)} \in \mathbb{R}^{(3n+m)^2 \times nm} \\
\frac{\partial \text{vec}(G_6)}{\partial \text{vec}(Y)} &= (\Lambda_1 \otimes I_{3n+m}) \cdot \frac{\partial \text{vec}(F_1)}{\partial \text{vec}(Y)} \in \mathbb{R}^{(3n+m)^2 \times n^2} \\
\frac{\partial \text{vec}(G_6)}{\partial \alpha} &= -(\Lambda_1 \otimes I_{3n+m}) \cdot \text{vec}(F_2) = -\text{vec}(\Lambda_1 F_2) \in \mathbb{R}^{(3n+m)^2 \times 1} \\
\frac{\partial \text{vec}(G_6)}{\partial \beta} &= -\text{vec}(\Lambda_1 M_{11}) \in \mathbb{R}^{(3n+m)^2 \times 1} \\
\frac{\partial \text{vec}(G_6)}{\partial \Delta} &= -\text{vec}(\Lambda_1 M_{11,B}) \in \mathbb{R}^{(3n+m)^2 \times 1} \\
\frac{\partial \text{vec}(G_6)}{\partial \text{vec}(\Lambda_3)} &= 0_{(3n+m)^2 \times (n+m)^2} \\
\frac{\partial \text{vec}(G_6)}{\partial \Lambda_\alpha} &= 0_{(3n+m)^2 \times 1} \\
\frac{\partial \text{vec}(G_6)}{\partial \Lambda_\beta} &= 0_{(3n+m)^2 \times 1} \\
\frac{\partial \text{vec}(G_6)}{\partial \Lambda_\Delta} &= 0_{(3n+m)^2 \times 1} \\
\frac{\partial \text{vec}(G_6)}{\partial \text{vec}(\Lambda_Y)} &= 0_{(3n+m)^2 \times n^2}
\end{align}

\textbf{注意:} $\frac{\partial \text{vec}(G_6)}{\partial \alpha}$の導出:
\begin{align}
\frac{\partial}{\partial \alpha} [(F_1 - \alpha F_2) \Lambda_1^T] &= \frac{\partial}{\partial \alpha} [F_1 - \alpha F_2] \cdot \Lambda_1^T \\
&= (-F_2) \Lambda_1^T = -\Lambda_1 F_2
\end{align}

したがって、$\frac{\partial \text{vec}(G_6)}{\partial \alpha} = -\text{vec}(\Lambda_1 F_2)$となる。同様に、$\beta$と$\Delta$についても$\Lambda_1$を左側に掛ける。

\subsection{$G_7$: 相補性条件 $F_3 \Lambda_3^T = 0$}

\begin{align}
G_7 = F_3 \Lambda_3^T = 0_{(n+m) \times (n+m)}
\end{align}

導出($D$で微分):

TABLE Vより、$F = AXB$のとき$\frac{\partial \text{vec}(F)}{\partial \text{vec}(X)} = B^T \otimes A$である。

$G_7 = F_3 \Lambda_3^T$について、$A = F_3$、$X = \Lambda_3^T$、$B = I_{n+m}$と見ると:
\begin{align}
\frac{\partial \text{vec}(F_3 \Lambda_3^T)}{\partial \text{vec}(\Lambda_3^T)} &= I_{n+m} \otimes F_3
\end{align}

さらに、$\Lambda_3^T$についてTABLE Vより:
\begin{align}
\frac{\partial \text{vec}(\Lambda_3^T)}{\partial \text{vec}(\Lambda_3)} &= K_{n+m,n+m}
\end{align}

したがって:
\begin{align}
\frac{\partial \text{vec}(G_7)}{\partial \text{vec}(\Lambda_3)} &= (I_{n+m} \otimes F_3) K_{n+m,n+m} \\
&= K_{n+m,n+m} (I_{n+m} \otimes F_3) \in \mathbb{R}^{(n+m)^2 \times (n+m)^2}
\end{align}

(注:実際のコードでは$K_{n+m,n+m} (I_{n+m} \otimes F_3)$の形式を使用)

次に、$F_3$に関する微分を考える。$G_7 = F_3 \Lambda_3^T$について、$A = \Lambda_3^T$、$X = F_3$、$B = I_{n+m}$と見ると:
\begin{align}
\frac{\partial \text{vec}(F_3 \Lambda_3^T)}{\partial \text{vec}(F_3)} &= I_{n+m} \otimes \Lambda_3^T = \Lambda_3 \otimes I_{n+m}
\end{align}

したがって:
\begin{align}
\frac{\partial \text{vec}(G_7)}{\partial \text{vec}(L)} &= (\Lambda_3 \otimes I_{n+m}) \cdot \frac{\partial \text{vec}(F_3)}{\partial \text{vec}(L)} \in \mathbb{R}^{(n+m)^2 \times nm} \\
\frac{\partial \text{vec}(G_7)}{\partial \text{vec}(Y)} &= (\Lambda_3 \otimes I_{n+m}) \cdot \frac{\partial \text{vec}(F_3)}{\partial \text{vec}(Y)} \in \mathbb{R}^{(n+m)^2 \times n^2} \\
\frac{\partial \text{vec}(G_7)}{\partial \alpha} &= 0_{(n+m)^2 \times 1} \\
\frac{\partial \text{vec}(G_7)}{\partial \beta} &= 0_{(n+m)^2 \times 1} \\
\frac{\partial \text{vec}(G_7)}{\partial \Delta} &= 0_{(n+m)^2 \times 1} \\
\frac{\partial \text{vec}(G_7)}{\partial \text{vec}(\Lambda_1)} &= 0_{(n+m)^2 \times (3n+m)^2} \\
\frac{\partial \text{vec}(G_7)}{\partial \Lambda_\alpha} &= 0_{(n+m)^2 \times 1} \\
\frac{\partial \text{vec}(G_7)}{\partial \Lambda_\beta} &= 0_{(n+m)^2 \times 1} \\
\frac{\partial \text{vec}(G_7)}{\partial \Lambda_\Delta} &= 0_{(n+m)^2 \times 1} \\
\frac{\partial \text{vec}(G_7)}{\partial \text{vec}(\Lambda_Y)} &= 0_{(n+m)^2 \times n^2}
\end{align}

\subsection{$G_8$: 相補性条件 $\alpha \Lambda_\alpha = 0$}

\begin{align}
G_8 = \alpha \Lambda_\alpha = 0
\end{align}

導出:
\begin{align}
\frac{\partial G_8}{\partial \text{vec}(L)} &= 0_{1 \times nm} \\
\frac{\partial G_8}{\partial \text{vec}(Y)} &= 0_{1 \times n^2} \\
\frac{\partial G_8}{\partial \alpha} &= \Lambda_\alpha \\
\frac{\partial G_8}{\partial \beta} &= 0 \\
\frac{\partial G_8}{\partial \Delta} &= 0 \\
\frac{\partial G_8}{\partial \text{vec}(\Lambda_1)} &= 0_{1 \times (3n+m)^2} \\
\frac{\partial G_8}{\partial \text{vec}(\Lambda_3)} &= 0_{1 \times (n+m)^2} \\
\frac{\partial G_8}{\partial \Lambda_\alpha} &= \alpha \\
\frac{\partial G_8}{\partial \Lambda_\beta} &= 0 \\
\frac{\partial G_8}{\partial \Lambda_\Delta} &= 0 \\
\frac{\partial G_8}{\partial \text{vec}(\Lambda_Y)} &= 0_{1 \times n^2}
\end{align}

\subsection{$G_9$: 相補性条件 $\beta \Lambda_\beta = 0$}

\begin{align}
G_9 = \beta \Lambda_\beta = 0
\end{align}

導出:
\begin{align}
\frac{\partial G_9}{\partial \text{vec}(L)} &= 0_{1 \times nm} \\
\frac{\partial G_9}{\partial \text{vec}(Y)} &= 0_{1 \times n^2} \\
\frac{\partial G_9}{\partial \alpha} &= 0 \\
\frac{\partial G_9}{\partial \beta} &= \Lambda_\beta \\
\frac{\partial G_9}{\partial \Delta} &= 0 \\
\frac{\partial G_9}{\partial \text{vec}(\Lambda_1)} &= 0_{1 \times (3n+m)^2} \\
\frac{\partial G_9}{\partial \text{vec}(\Lambda_3)} &= 0_{1 \times (n+m)^2} \\
\frac{\partial G_9}{\partial \Lambda_\alpha} &= 0 \\
\frac{\partial G_9}{\partial \Lambda_\beta} &= \beta \\
\frac{\partial G_9}{\partial \Lambda_\Delta} &= 0 \\
\frac{\partial G_9}{\partial \text{vec}(\Lambda_Y)} &= 0_{1 \times n^2}
\end{align}

\subsection{$G_{10}$: 相補性条件 $\Delta \Lambda_\Delta = 0$}

\begin{align}
G_{10} = \Delta \Lambda_\Delta = 0
\end{align}

導出:
\begin{align}
\frac{\partial G_{10}}{\partial \text{vec}(L)} &= 0_{1 \times nm} \\
\frac{\partial G_{10}}{\partial \text{vec}(Y)} &= 0_{1 \times n^2} \\
\frac{\partial G_{10}}{\partial \alpha} &= 0 \\
\frac{\partial G_{10}}{\partial \beta} &= 0 \\
\frac{\partial G_{10}}{\partial \Delta} &= \Lambda_\Delta \\
\frac{\partial G_{10}}{\partial \text{vec}(\Lambda_1)} &= 0_{1 \times (3n+m)^2} \\
\frac{\partial G_{10}}{\partial \text{vec}(\Lambda_3)} &= 0_{1 \times (n+m)^2} \\
\frac{\partial G_{10}}{\partial \Lambda_\alpha} &= 0 \\
\frac{\partial G_{10}}{\partial \Lambda_\beta} &= 0 \\
\frac{\partial G_{10}}{\partial \Lambda_\Delta} &= \Delta \\
\frac{\partial G_{10}}{\partial \text{vec}(\Lambda_Y)} &= 0_{1 \times n^2}
\end{align}

\subsection{$G_{11}$: 対称性条件 $\Lambda_1 - \Lambda_1^T = 0$}

\begin{align}
G_{11} = \Lambda_1 - \Lambda_1^T = 0_{(3n+m) \times (3n+m)}
\end{align}

導出:
\begin{align}
\frac{\partial \text{vec}(G_{11})}{\partial \text{vec}(L)} &= 0_{(3n+m)^2 \times nm} \\
\frac{\partial \text{vec}(G_{11})}{\partial \text{vec}(Y)} &= 0_{(3n+m)^2 \times n^2} \\
\frac{\partial \text{vec}(G_{11})}{\partial \alpha} &= 0_{(3n+m)^2 \times 1} \\
\frac{\partial \text{vec}(G_{11})}{\partial \beta} &= 0_{(3n+m)^2 \times 1} \\
\frac{\partial \text{vec}(G_{11})}{\partial \Delta} &= 0_{(3n+m)^2 \times 1} \\
\frac{\partial \text{vec}(G_{11})}{\partial \text{vec}(\Lambda_1)} &= I_{(3n+m)^2} - K_{3n+m,3n+m} \in \mathbb{R}^{(3n+m)^2 \times (3n+m)^2} \\
\frac{\partial \text{vec}(G_{11})}{\partial \text{vec}(\Lambda_3)} &= 0_{(3n+m)^2 \times (n+m)^2} \\
\frac{\partial \text{vec}(G_{11})}{\partial \Lambda_\alpha} &= 0_{(3n+m)^2 \times 1} \\
\frac{\partial \text{vec}(G_{11})}{\partial \Lambda_\beta} &= 0_{(3n+m)^2 \times 1} \\
\frac{\partial \text{vec}(G_{11})}{\partial \Lambda_\Delta} &= 0_{(3n+m)^2 \times 1} \\
\frac{\partial \text{vec}(G_{11})}{\partial \text{vec}(\Lambda_Y)} &= 0_{(3n+m)^2 \times n^2}
\end{align}

\subsection{$G_{12}$: 対称性条件 $\Lambda_3 - \Lambda_3^T = 0$}

\begin{align}
G_{12} = \Lambda_3 - \Lambda_3^T = 0_{(n+m) \times (n+m)}
\end{align}

導出:
\begin{align}
\frac{\partial \text{vec}(G_{12})}{\partial \text{vec}(L)} &= 0_{(n+m)^2 \times nm} \\
\frac{\partial \text{vec}(G_{12})}{\partial \text{vec}(Y)} &= 0_{(n+m)^2 \times n^2} \\
\frac{\partial \text{vec}(G_{12})}{\partial \alpha} &= 0_{(n+m)^2 \times 1} \\
\frac{\partial \text{vec}(G_{12})}{\partial \beta} &= 0_{(n+m)^2 \times 1} \\
\frac{\partial \text{vec}(G_{12})}{\partial \Delta} &= 0_{(n+m)^2 \times 1} \\
\frac{\partial \text{vec}(G_{12})}{\partial \text{vec}(\Lambda_1)} &= 0_{(n+m)^2 \times (3n+m)^2} \\
\frac{\partial \text{vec}(G_{12})}{\partial \text{vec}(\Lambda_3)} &= I_{(n+m)^2} - K_{n+m,n+m} \in \mathbb{R}^{(n+m)^2 \times (n+m)^2} \\
\frac{\partial \text{vec}(G_{12})}{\partial \Lambda_\alpha} &= 0_{(n+m)^2 \times 1} \\
\frac{\partial \text{vec}(G_{12})}{\partial \Lambda_\beta} &= 0_{(n+m)^2 \times 1} \\
\frac{\partial \text{vec}(G_{12})}{\partial \Lambda_\Delta} &= 0_{(n+m)^2 \times 1} \\
\frac{\partial \text{vec}(G_{12})}{\partial \text{vec}(\Lambda_Y)} &= 0_{(n+m)^2 \times n^2}
\end{align}

\subsection{$G_{13}$: 対称性条件 $Y - Y^T = 0$}

\begin{align}
G_{13} = Y - Y^T = 0_{n \times n}
\end{align}

導出:
\begin{align}
\frac{\partial \text{vec}(G_{13})}{\partial \text{vec}(L)} &= 0_{n^2 \times nm} \\
\frac{\partial \text{vec}(G_{13})}{\partial \text{vec}(Y)} &= I_{n^2} - K_{n,n} \in \mathbb{R}^{n^2 \times n^2} \\
\frac{\partial \text{vec}(G_{13})}{\partial \alpha} &= 0_{n^2 \times 1} \\
\frac{\partial \text{vec}(G_{13})}{\partial \beta} &= 0_{n^2 \times 1} \\
\frac{\partial \text{vec}(G_{13})}{\partial \Delta} &= 0_{n^2 \times 1} \\
\frac{\partial \text{vec}(G_{13})}{\partial \text{vec}(\Lambda_1)} &= 0_{n^2 \times (3n+m)^2} \\
\frac{\partial \text{vec}(G_{13})}{\partial \text{vec}(\Lambda_3)} &= 0_{n^2 \times (n+m)^2} \\
\frac{\partial \text{vec}(G_{13})}{\partial \Lambda_\alpha} &= 0_{n^2 \times 1} \\
\frac{\partial \text{vec}(G_{13})}{\partial \Lambda_\beta} &= 0_{n^2 \times 1} \\
\frac{\partial \text{vec}(G_{13})}{\partial \Lambda_\Delta} &= 0_{n^2 \times 1} \\
\frac{\partial \text{vec}(G_{13})}{\partial \text{vec}(\Lambda_Y)} &= 0_{n^2 \times n^2}
\end{align}

\textbf{注意:} $I_{n^2} - K_{n,n}$のrankは$n(n+1)/2$である。これは対称行列$Y$の独立な要素の数に対応する。

\subsection{$G_{14}$: 相補性条件 $Y \Lambda_Y^T = 0$}

\begin{align}
G_{14} = Y \Lambda_Y^T = 0_{n \times n}
\end{align}

導出($D$で微分):

TABLE Vより、$F = AXB$のとき$\frac{\partial \text{vec}(F)}{\partial \text{vec}(X)} = B^T \otimes A$である。

$G_{14} = Y \Lambda_Y^T$について、$A = Y$、$X = \Lambda_Y^T$、$B = I_n$と見ると:
\begin{align}
\frac{\partial \text{vec}(Y \Lambda_Y^T)}{\partial \text{vec}(\Lambda_Y^T)} &= I_n \otimes Y
\end{align}

さらに、$\Lambda_Y^T$についてTABLE Vより:
\begin{align}
\frac{\partial \text{vec}(\Lambda_Y^T)}{\partial \text{vec}(\Lambda_Y)} &= K_{n,n}
\end{align}

したがって:
\begin{align}
\frac{\partial \text{vec}(G_{14})}{\partial \text{vec}(\Lambda_Y)} &= (I_n \otimes Y) K_{n,n} \\
&= K_{n,n} (I_n \otimes Y) \in \mathbb{R}^{n^2 \times n^2}
\end{align}

(注:実際のコードでは$K_{n,n} (I_n \otimes Y)$の形式を使用)

次に、$Y$に関する微分を考える。$G_{14} = Y \Lambda_Y^T$について、$A = \Lambda_Y^T$、$X = Y$、$B = I_n$と見ると:
\begin{align}
\frac{\partial \text{vec}(Y \Lambda_Y^T)}{\partial \text{vec}(Y)} &= I_n \otimes \Lambda_Y^T = \Lambda_Y \otimes I_n
\end{align}

したがって:
\begin{align}
\frac{\partial \text{vec}(G_{14})}{\partial \text{vec}(L)} &= 0_{n^2 \times nm} \\
\frac{\partial \text{vec}(G_{14})}{\partial \text{vec}(Y)} &= \Lambda_Y \otimes I_n \in \mathbb{R}^{n^2 \times n^2} \\
\frac{\partial \text{vec}(G_{14})}{\partial \alpha} &= 0_{n^2 \times 1} \\
\frac{\partial \text{vec}(G_{14})}{\partial \beta} &= 0_{n^2 \times 1} \\
\frac{\partial \text{vec}(G_{14})}{\partial \Delta} &= 0_{n^2 \times 1} \\
\frac{\partial \text{vec}(G_{14})}{\partial \text{vec}(\Lambda_1)} &= 0_{n^2 \times (3n+m)^2} \\
\frac{\partial \text{vec}(G_{14})}{\partial \text{vec}(\Lambda_3)} &= 0_{n^2 \times (n+m)^2} \\
\frac{\partial \text{vec}(G_{14})}{\partial \Lambda_\alpha} &= 0_{n^2 \times 1} \\
\frac{\partial \text{vec}(G_{14})}{\partial \Lambda_\beta} &= 0_{n^2 \times 1} \\
\frac{\partial \text{vec}(G_{14})}{\partial \Lambda_\Delta} &= 0_{n^2 \times 1}
\end{align}

\textbf{注意:} $\frac{\partial}{\partial \Lambda_Y} [Y \Lambda_Y^T] = Y$であるため、$\frac{\partial \text{vec}(G_{14})}{\partial \text{vec}(\Lambda_Y)} = (I_n \otimes Y) K_{n,n}$となる。

\subsection{$G_{15}$: 対称性条件 $\Lambda_Y - \Lambda_Y^T = 0$}

\begin{align}
G_{15} = \Lambda_Y - \Lambda_Y^T = 0_{n \times n}
\end{align}

導出:
\begin{align}
\frac{\partial \text{vec}(G_{15})}{\partial \text{vec}(L)} &= 0_{n^2 \times nm} \\
\frac{\partial \text{vec}(G_{15})}{\partial \text{vec}(Y)} &= 0_{n^2 \times n^2} \\
\frac{\partial \text{vec}(G_{15})}{\partial \alpha} &= 0_{n^2 \times 1} \\
\frac{\partial \text{vec}(G_{15})}{\partial \beta} &= 0_{n^2 \times 1} \\
\frac{\partial \text{vec}(G_{15})}{\partial \Delta} &= 0_{n^2 \times 1} \\
\frac{\partial \text{vec}(G_{15})}{\partial \text{vec}(\Lambda_1)} &= 0_{n^2 \times (3n+m)^2} \\
\frac{\partial \text{vec}(G_{15})}{\partial \text{vec}(\Lambda_3)} &= 0_{n^2 \times (n+m)^2} \\
\frac{\partial \text{vec}(G_{15})}{\partial \Lambda_\alpha} &= 0_{n^2 \times 1} \\
\frac{\partial \text{vec}(G_{15})}{\partial \Lambda_\beta} &= 0_{n^2 \times 1} \\
\frac{\partial \text{vec}(G_{15})}{\partial \Lambda_\Delta} &= 0_{n^2 \times 1} \\
\frac{\partial \text{vec}(G_{15})}{\partial \text{vec}(\Lambda_Y)} &= I_{n^2} - K_{n,n} \in \mathbb{R}^{n^2 \times n^2}
\end{align}

\newpage
\section{線形システムの構築}

すべての制約をまとめると、以下の線形システムが得られる:

\begin{align}
H \begin{bmatrix}
\frac{d\text{vec}(L)}{d\text{vec}(D)} \\
\frac{d\text{vec}(Y)}{d\text{vec}(D)} \\
\frac{d\alpha}{d\text{vec}(D)} \\
\frac{d\beta}{d\text{vec}(D)} \\
\frac{d\Delta}{d\text{vec}(D)} \\
\frac{d\text{vec}(\Lambda_1)}{d\text{vec}(D)} \\
\frac{d\text{vec}(\Lambda_3)}{d\text{vec}(D)} \\
\frac{d\Lambda_\alpha}{d\text{vec}(D)} \\
\frac{d\Lambda_\beta}{d\text{vec}(D)} \\
\frac{d\Lambda_\Delta}{d\text{vec}(D)} \\
\frac{d\text{vec}(\Lambda_Y)}{d\text{vec}(D)}
\end{bmatrix} = -\begin{bmatrix}
\frac{\partial G_5}{\partial \text{vec}(D)} \\
\frac{\partial G_1}{\partial \text{vec}(D)} \\
\frac{\partial G_2}{\partial \text{vec}(D)} \\
\frac{\partial G_3}{\partial \text{vec}(D)} \\
\frac{\partial G_4}{\partial \text{vec}(D)} \\
\frac{\partial G_6}{\partial \text{vec}(D)} \\
\frac{\partial G_7}{\partial \text{vec}(D)} \\
\frac{\partial G_8}{\partial \text{vec}(D)} \\
\frac{\partial G_9}{\partial \text{vec}(D)} \\
\frac{\partial G_{10}}{\partial \text{vec}(D)} \\
\frac{\partial G_{11}}{\partial \text{vec}(D)} \\
\frac{\partial G_{12}}{\partial \text{vec}(D)} \\
\frac{\partial G_{13}}{\partial \text{vec}(D)} \\
\frac{\partial G_{14}}{\partial \text{vec}(D)} \\
\frac{\partial G_{15}}{\partial \text{vec}(D)}
\end{bmatrix}
\end{align}

ここで、$H$は各$\frac{\partial G_i}{\partial \bm{x}}$を縦に並べた行列である。各$\frac{\partial G_i}{\partial \text{vec}(D)}$は行列であり、右辺全体$B$も行列である。

\section{勾配の計算}

線形システムを解くことで、$\frac{d\Delta}{d\text{vec}(D)}$が得られる。これは目的関数の勾配に対応する。実際の使用では、$\frac{d\Delta}{d\text{vec}(D)}$を$\mathbb{R}^{(2n+m) \times T}$の形状にreshapeして、$\frac{d\Delta}{dD}$として扱う。

\end{document}
